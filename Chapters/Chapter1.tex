\chapter{总述} % Main chapter title

\label{Chapter1} % Change X to a consecutive number; for referencing this chapter elsewhere, use \ref{ChapterX}

架构支撑的操作系统安全增强
安全,一直是操作系统领域的一个重要方向。
一直以来,研究者们一直致力于保护数据的安全,
保证执行的可靠,
避免代码的 bug。
而由架构支撑的操作系统安全增强,则是其中非常有价值的一个方向。
安全与性能往往是需要权衡的,
加入更多的校验操作,
必然会使程序的运行变得缓慢。
启用硬件架构支撑的安全特性而非纯软件的实现,
可以大大提高安全校验操作的性能,
极大降低提高安全性所带来的开销。

%----------------------------------------------------------------------------------------
\section{硬件扩展的发展}
%----------------------------------------------------------------------------------------
近年来,不断有新的硬件支持的操作系统安全领域的技术出现。
随着技术的不断迭代,硬件扩展的安全技术向着更安全、更方便、更高效不断发展。
例如:
\begin{itemize}
    \item SMEP(Supervisor Mode Execution Prevention) 2011年
    \item SMAP(Supervisor Mode Access Prevention) 2012年
    \item MPX(Memory Protection eXtension)2013年
    \item MPK(Memory Protection Keys)2015年
    \item ARM Pointer Authentication 2017年
    \item Intel Control-flow Enforcement Technology 2018年
\end{itemize}
近年来,TEE 可信执行环境受到了学者们的关注。
在云服务器逐渐被用户接纳的当下,
如何让用户放心大胆的将自己的敏感数据与敏感计算放到云上去运行成为了难题。
TEE 的出现正能回答这个问题。
云服务提供商在自己的硬件设备中创建一个隔离的执行环境,
通过一系列手段使用户相信,在这个环境中执行的应用一定不会被篡改,
数据一定是安全的。
由此使用户敢于将敏感应用云端化,使用云端资源。
而如何构建一个可信执行环境 TEE,不同的硬件架构提供商给出了不同的解决方案。

%--------------------------------------
\subsection{ARM TrustZone}
%--------------------------------------
TODO
%--------------------------------------
\subsection{Graphene-SGX}
%--------------------------------------
\paragraph{Intel SGX}Intel SGX技术通过硬件扩展,提供了 enclave 以及其相关的一些特性。
用户态应用程序可以部分或全部的放入 enclave执行,SGX技术可以保证,
在enclave中执行的程序不会被恶意的操作系统或恶意的Hypervisor攻击,
且数据不会泄露。

%----------------------------------------------------------------------------------------
\subsection{RISC-V KeyStone}
%----------------------------------------------------------------------------------------
TODO

