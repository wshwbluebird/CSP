\chapter{总述} % Main chapter title

\label{Chapter1} % Change X to a consecutive number; for referencing this chapter elsewhere, use \ref{ChapterX}

架构支撑的操作系统安全增强
安全,一直是操作系统领域的一个重要方向。
一直以来,研究者们一直致力于保护数据的安全,
保证执行的可靠,
避免代码的 bug。
而由架构支撑的操作系统安全增强,则是其中非常有价值的一个方向。
安全与性能往往是需要权衡的,
加入更多的校验操作,
必然会使程序的运行变得缓慢。
启用硬件架构支撑的安全特性而非纯软件的实现,
可以大大提高安全校验操作的性能,
极大降低提高安全性所带来的开销。

%----------------------------------------------------------------------------------------
\section{硬件扩展的发展}
%----------------------------------------------------------------------------------------
近年来,不断有新的硬件支持的操作系统安全领域的技术出现。
随着技术的不断迭代,硬件扩展的安全技术向着更安全、更方便、更高效不断发展。
例如:
\begin{itemize}
    \item SMEP(Supervisor Mode Execution Prevention) 2011年
    \item SMAP(Supervisor Mode Access Prevention) 2012年
    \item MPX(Memory Protection eXtension)2013年
    \item MPK(Memory Protection Keys)2015年
    \item ARM Pointer Authentication 2017年
    \item Intel Control-flow Enforcement Technology 2018年
\end{itemize}
近年来,TEE 可信执行环境受到了学者们的关注。
在云服务器逐渐被用户接纳的当下,
如何让用户放心大胆的将自己的敏感数据与敏感计算放到云上去运行成为了难题。
TEE 的出现正能回答这个问题。
云服务提供商在自己的硬件设备中创建一个隔离的执行环境,
通过一系列手段使用户相信,在这个环境中执行的应用一定不会被篡改,
数据一定是安全的。
由此使用户敢于将敏感应用云端化,使用云端资源。
而如何构建一个可信执行环境 TEE,不同的硬件架构提供商给出了不同的解决方案。

%--------------------------------------
\subsection{ARM TrustZone}
%--------------------------------------
\paragraph{Normal World 和 Secure World}TrustZone通过TrustedFirmware这一硬件扩展,
将计算机资源划分为安全世界与普通世界。
同时带来与Intel不同的 EL0-EL3 这4个特权等级,
EL3被称为 monitor mode,
只有monitor mode 可以通过运行 ARM Trusted Firmware 来切换普通世界与安全世界。
一台有 TrustZone 的机器,
会经过安全的 bootloader 校验并启动安全世界中的操作系统,
而后,再切换到不可信的普通世界去执行。
由于安全世界严格的启动逻辑,可以保证启动的安全世界中的 kernel 没有被修改过,
是可信的。
\paragraph{SANCTUARY}TrustZone两种世界的划分也会带来一些问题,
例如安全世界内的应用一旦被攻破,能轻易的影响安全世界中的其他敏感数据。
因此,保证安全世界内的应用可信变得由为重要,
这直接导致应用开发者的工作量增大,技术适配困难。
为此SANCTUARY提出了普通世界与安全世界的中间层 Sanctuary的概念。
并区分可信应用程序和敏感应用程序,
使TrustZone对软件的可信要求降低,
一定程度上解决了运营商与应用开发者之间的协调问题。

%--------------------------------------
\subsection{Graphene-SGX}
%--------------------------------------
\paragraph{Intel SGX}Intel SGX技术通过硬件扩展,提供了 enclave 以及其相关的一些特性。
用户态应用程序可以部分或全部的放入 enclave执行,SGX技术可以保证,
在enclave中执行的程序不会被恶意的操作系统或恶意的Hypervisor攻击,
且数据不会泄露。
\paragraph{Haven}不过,
应用程序适配 SGX是一个挑战。
由于安全性的考虑,
encalve中的程序有一些操作不被允许(例如不能使用syscall),
因此,已有的应用程序想要利用 SGX 带来的安全性,
需要付出大量的重构努力。
OSDI 14 的一个叫 Haven 的工作,提出了带壳执行,证明了
可以将一个已有的二进制应用程序,
通过 libOS 的方式,整个放到enclave中执行。
通过对libOS的API实现进行修改,将底层实现缩小到更少的系统调用,
减少了攻击面。
再对这些与enclave外不受信任的操作系统或 hypervisor 沟通的方法进行校验。
保证出入enclave的控制流的可靠性,并对传输数据进行加密,
已有的应用程序就可以通过把整个二进制放入enclave中的方式利用SGX技术提供的安全性。
\paragraph{Graphene-SGX}Graphene-SGX在 Haven的基础上更进一步,
实现了enclave的安全fork与可信通信,
极大的提高了现有应用利用SGX技术时的可扩展性。
同时,由于硬件的发展,Graphene-SGX的测试还证明了
libOS化的应用程序放入enclave中所带来的开销是可以接受的。

%----------------------------------------------------------------------------------------
\subsection{RISC-V KeyStone}
%----------------------------------------------------------------------------------------
\paragraph{RISC-V}
RISC-V 是一个基于精简指令集(RISC)原则的开源指令集架构(ISA),简易解释为开源软件运动相对应的一种“开源硬件”。该项目2010年始于加州大学柏克莱分校,但许多贡献者是该大学以外的志愿者和行业工作者。
与大多数指令集相比,RISC-V指令集可以自由地用于任何目的,允许任何人设计、制造和销售RISC-V芯片和软件而不必支付给任何公司专利费。
RISC-V指令集的设计考虑了小型、快速、低功耗的现实情况来实做,但并没有对特定的微架构做过度的设计。
\paragraph{Physical Memory Protection}
PMP(Physical Memory Protection)RISC-V 提供的物理内存隔离机制。
RISC-V 拥有16个 PMPADDR 寄存器和4个 PMPCFG 寄存器(64位下只使用其中两个)。地址寄存器和配置寄存器共同决定了16个 PMP region。
PMP region 可以视为 enclave,每个线程只能访问自己所拥有的 region,以此实现物理内存的隔离。
\paragraph{Keystone}
Keystone 是第一款构建定制 TEE 的开源框架。在 Keystone 中,enclave 使用 PMP 进行隔离,并可选择对 PMP region 做 cache line binding,每个 region 只能访问特定的一个或多个 cache line,以此防御 cache side-channel attack。
Keystone 还自带了一个可信的 RunTime。这个 RunTime 运行在 Supervisor-mode,可以动态增减 enclave 占用的内存大小,并且可以处理 enclave 的 page fault。
Keystone 在 RISC-V 的基础上提供上述的安全扩展,可以使应用快速、安全地部署在 RISC-V 平台之上。