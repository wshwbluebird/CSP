\label{Chapter5} % Change X to a consecutive number; for referencing this chapter elsewhere, use \ref{ChapterX}
\chapter{扩展方案} % Main chapter title
\section{目标问题}
上述提到给予RISC-V中PMP设计的KeyStone存在PMP寄存器不足的问题,
如(图~\ref{fig:pmpcfg}), 在RV64的模式下有2个pmpcfg寄存器pmpcfg0和pmpcfg2,
包含了可以指明解析方案的16个8个bits组成的pmpxcfg寄存器,即pmp0cfg-pmp15cfg。
在一些场景下仅仅这16个权限寄存器可能并不足够。为了在不改变原始硬件的情况下,
增加可用的PMP寄存器范围,可以给出PMP寄存器的扩展方案。

\begin{figure}
    \centering
    \includegraphics[scale=0.30]{Figures/extend/pmpcfg.png}
    \decoRule
    \caption{pmpcfg寄存器}
    \label{fig:pmpcfg}
\end{figure}

\begin{figure}
    \centering
    \includegraphics[scale=0.30]{Figures/extend/addrformat.png}
    \decoRule
    \caption{地址寄存器格式}
    \label{fig:addrformat}
\end{figure}

\section{扩展思路}
原始的PMP是通过Monitor来配置Machine Code用来保护Monitor最基本的物理地址空间。
为了扩大保护的范围,可以考虑当一个Kernel已经进入了由PMP寄存器保护的物理内存之后,即附属在一个hart上后,
可以借用原来的寄存器在Kernel层再做一次映射,以此从物理上在同一个kernel中隔离更多的空间。
当一个用户端的App从U状态下进行向下寻址后,
可以先由PMA检查和原始PMP检查对其进行校验,在通过之后,可以再使用Kernel层上的扩展隔离,
以此达到进一步划分物理空间和App所属的地址隔离问题,CFG基址寄存器如(图~\ref{fig:addrformat})。

\subsection{实际扩展}
具体的扩展方案,可以通过一个线程在某一Kernel由PMP分配的物理空间之后,即hart运行在U模式下时,
可以将PMP寄存器进行重置和复用,可以通过U模式下重置,可以在U模式下用和PMP相同的策略进一步划分内存。
首先hart进入U模式,Machine Code将原始PMP寄存器的值保存至Security Monitor中,
之后使用Kernel自己的策略重置PMP寄存器,进行地址的划分和隔离。
当离开U模式时,Machine Code也会恢复原来存储在Security Monitor中的寄存器的值。



\subsection{扩展格式}
在U模式下的扩展采用原有PMP的隔离策略,并在异常等方面做降级处理。
仍然采用如(图~\ref{fig:pmpcfg})的方式,使用pmpcfg0和pmpfcg2寄存器存储pmp0cfg到pmp15cfg的值,来划分权限,使用PMP地址寄存器来存储地址划分的基础地址。
CFG的格式和原始PMP积寄存器CFG基本一致,如(图~\ref{fig:cfgformat}),X位,W位和R位分别表示可读可写和可执行的权限。
A位表示寻址策略,如(图~\ref{fig:afield})所示,当A的值为0时,表示此扩展禁用不匹配任何地址。当A=1时采用TOR分配到地址空间的上界。
当A=2时采用NA4策略,分配4byte对齐的空间。
当A=3时采用NAPOT策略,即分配一个对其且大小为2的整数幂的空间,在这种模式下最小分配地址为8bytes。
划分地址空间的结果(图~\ref{fig:address})
L位代表着对新的PMP扩展进行上锁。以上的权限只在U模式下生效,原来的M模式仍然可以正常的访问内存,
且可以重置和解锁扩展PMP的L位。
\begin{figure}
    \centering
    \includegraphics[scale=0.30]{Figures/extend/cfgformat.png}
    \decoRule
    \caption{CFG格式}
    \label{fig:cfgformat}
\end{figure}
\begin{figure}
    \centering
    \includegraphics[scale=0.40]{Figures/extend/afield.png}
    \decoRule
    \caption{A位取值}
    \label{fig:afield}
\end{figure}
\begin{figure}
    \centering
    \includegraphics[scale=0.40]{Figures/extend/address.png}
    \decoRule
    \caption{地址空间划分}
    \label{fig:address}
\end{figure}

\subsection{权限和异常}
扩展的PMP只会在寻址已经通过了原来PMP的检查之后才会进行。
原始的PMP会通过最低位表示一次访问存储的成功或者失败。
首先PMP会检测其访问的地址是否在访问范围内,如果在不在访问范围内,则直接将结果设置为失败。
若在访问范围内,则根据实际的读写执行权限进行判定。
若访问失败,则会出现一个指令访问异常。
和原始PMP类似扩展的PMP也使用最低位来表示访存的失败,在原来PMP的检查之后,U模式下仍然会先检查方寸是否在扩展PMP规定的范围内,
这次检查相比于原始PMP的检查具有更细的粒度。在检查通过之后,仍然会根据读写执行权限进行判定,
若访问失败,会产生出一个页错误,这个错误可以交给U模式下kernel自己处理,也可以让更高层的Machine Code进行处理。


\section{场景分析}
\subsection{适用场景}
\subsection{不适用场景}

\section{小结}