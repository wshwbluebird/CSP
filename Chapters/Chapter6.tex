\chapter{总结} % Main chapter title

\label{Chapter6} % Change X to a consecutive number; for referencing this chapter elsewhere, use \ref{ChapterX}
本篇调研从题目出发,首先对ARM TrustZone,Intel SGX和RISC-V这一类硬件扩展及指令集进行了调研学习,列出了其基本设计思想和硬件实现思路
之后对于每一种特殊的硬件扩展,分别选用了一种Case进行样例研究,具体包括基于ARM TrustZone的SANCTUARY,基于Intel SGX的Graphene,基于
RISC-V的KeyStone设计。对于每一种Case,分别列出了威胁模型,保证的安全性质和能够防御的一些具体攻击,分析了具体设计的性能,
可扩展性和不足之处。

对于基于ARM TrustZone的SANCTUARY,除了给出了具体的分析之外,还列出了不适合SANCTUARY使用的具体场景,并对其不能支持多线程的问题在
视频渲染场景下的性能开销问题进行具体分析。对于基于RISC-V的KeyStone设计,我们在PMP方案原本的基础上,提出了基于PMP隔离不充分的问题,给出扩展方案来优化在上述分析中的不足点。主要解决了PMP寄存器
数量较少和在IoT场景中,没有页表情况下,只有PMP隔离不充足的问题。最后给出了具体的场景分析,来说设计的实用性。

总结来说系统安全的重要性随着人们越来越多的使用计算资源和依赖系统平台而变得越来越重要,而很多单纯依靠软件解决的安全问题主要
将面临威胁解决不彻底等问题,其本质也是性能与安全性的折衷,很大程度上牺牲了性能,很多设计具有安全性,但是一定程度上不具有可用性。
而硬件的解决方案,虽然有性能上的优势,但是由于是固件很难具有通用性。
在如今如数字版权保护,个人隐私保护,数据加密安全等方面都需要系统安全措施,所以基于硬件辅助和软件扩展的安全解决方案也必将成为主流。